% Options for packages loaded elsewhere
% Options for packages loaded elsewhere
\PassOptionsToPackage{unicode}{hyperref}
\PassOptionsToPackage{hyphens}{url}
\PassOptionsToPackage{dvipsnames,svgnames,x11names}{xcolor}
%
\documentclass[
  letterpaper,
  DIV=11,
  numbers=noendperiod]{scrreprt}
\usepackage{xcolor}
\usepackage{amsmath,amssymb}
\setcounter{secnumdepth}{5}
\usepackage{iftex}
\ifPDFTeX
  \usepackage[T1]{fontenc}
  \usepackage[utf8]{inputenc}
  \usepackage{textcomp} % provide euro and other symbols
\else % if luatex or xetex
  \usepackage{unicode-math} % this also loads fontspec
  \defaultfontfeatures{Scale=MatchLowercase}
  \defaultfontfeatures[\rmfamily]{Ligatures=TeX,Scale=1}
\fi
\usepackage{lmodern}
\ifPDFTeX\else
  % xetex/luatex font selection
\fi
% Use upquote if available, for straight quotes in verbatim environments
\IfFileExists{upquote.sty}{\usepackage{upquote}}{}
\IfFileExists{microtype.sty}{% use microtype if available
  \usepackage[]{microtype}
  \UseMicrotypeSet[protrusion]{basicmath} % disable protrusion for tt fonts
}{}
\makeatletter
\@ifundefined{KOMAClassName}{% if non-KOMA class
  \IfFileExists{parskip.sty}{%
    \usepackage{parskip}
  }{% else
    \setlength{\parindent}{0pt}
    \setlength{\parskip}{6pt plus 2pt minus 1pt}}
}{% if KOMA class
  \KOMAoptions{parskip=half}}
\makeatother
% Make \paragraph and \subparagraph free-standing
\makeatletter
\ifx\paragraph\undefined\else
  \let\oldparagraph\paragraph
  \renewcommand{\paragraph}{
    \@ifstar
      \xxxParagraphStar
      \xxxParagraphNoStar
  }
  \newcommand{\xxxParagraphStar}[1]{\oldparagraph*{#1}\mbox{}}
  \newcommand{\xxxParagraphNoStar}[1]{\oldparagraph{#1}\mbox{}}
\fi
\ifx\subparagraph\undefined\else
  \let\oldsubparagraph\subparagraph
  \renewcommand{\subparagraph}{
    \@ifstar
      \xxxSubParagraphStar
      \xxxSubParagraphNoStar
  }
  \newcommand{\xxxSubParagraphStar}[1]{\oldsubparagraph*{#1}\mbox{}}
  \newcommand{\xxxSubParagraphNoStar}[1]{\oldsubparagraph{#1}\mbox{}}
\fi
\makeatother


\usepackage{longtable,booktabs,array}
\usepackage{calc} % for calculating minipage widths
% Correct order of tables after \paragraph or \subparagraph
\usepackage{etoolbox}
\makeatletter
\patchcmd\longtable{\par}{\if@noskipsec\mbox{}\fi\par}{}{}
\makeatother
% Allow footnotes in longtable head/foot
\IfFileExists{footnotehyper.sty}{\usepackage{footnotehyper}}{\usepackage{footnote}}
\makesavenoteenv{longtable}
\usepackage{graphicx}
\makeatletter
\newsavebox\pandoc@box
\newcommand*\pandocbounded[1]{% scales image to fit in text height/width
  \sbox\pandoc@box{#1}%
  \Gscale@div\@tempa{\textheight}{\dimexpr\ht\pandoc@box+\dp\pandoc@box\relax}%
  \Gscale@div\@tempb{\linewidth}{\wd\pandoc@box}%
  \ifdim\@tempb\p@<\@tempa\p@\let\@tempa\@tempb\fi% select the smaller of both
  \ifdim\@tempa\p@<\p@\scalebox{\@tempa}{\usebox\pandoc@box}%
  \else\usebox{\pandoc@box}%
  \fi%
}
% Set default figure placement to htbp
\def\fps@figure{htbp}
\makeatother





\setlength{\emergencystretch}{3em} % prevent overfull lines

\providecommand{\tightlist}{%
  \setlength{\itemsep}{0pt}\setlength{\parskip}{0pt}}



 


\KOMAoption{captions}{tableheading}
\makeatletter
\@ifpackageloaded{bookmark}{}{\usepackage{bookmark}}
\makeatother
\makeatletter
\@ifpackageloaded{caption}{}{\usepackage{caption}}
\AtBeginDocument{%
\ifdefined\contentsname
  \renewcommand*\contentsname{Table of contents}
\else
  \newcommand\contentsname{Table of contents}
\fi
\ifdefined\listfigurename
  \renewcommand*\listfigurename{List of Figures}
\else
  \newcommand\listfigurename{List of Figures}
\fi
\ifdefined\listtablename
  \renewcommand*\listtablename{List of Tables}
\else
  \newcommand\listtablename{List of Tables}
\fi
\ifdefined\figurename
  \renewcommand*\figurename{Figure}
\else
  \newcommand\figurename{Figure}
\fi
\ifdefined\tablename
  \renewcommand*\tablename{Table}
\else
  \newcommand\tablename{Table}
\fi
}
\@ifpackageloaded{float}{}{\usepackage{float}}
\floatstyle{ruled}
\@ifundefined{c@chapter}{\newfloat{codelisting}{h}{lop}}{\newfloat{codelisting}{h}{lop}[chapter]}
\floatname{codelisting}{Listing}
\newcommand*\listoflistings{\listof{codelisting}{List of Listings}}
\makeatother
\makeatletter
\makeatother
\makeatletter
\@ifpackageloaded{caption}{}{\usepackage{caption}}
\@ifpackageloaded{subcaption}{}{\usepackage{subcaption}}
\makeatother
\usepackage{bookmark}
\IfFileExists{xurl.sty}{\usepackage{xurl}}{} % add URL line breaks if available
\urlstyle{same}
\hypersetup{
  pdftitle={Jonathan Wiguna},
  pdfauthor={Jonathan Wiguna},
  colorlinks=true,
  linkcolor={blue},
  filecolor={Maroon},
  citecolor={Blue},
  urlcolor={Blue},
  pdfcreator={LaTeX via pandoc}}


\title{Jonathan Wiguna}
\usepackage{etoolbox}
\makeatletter
\providecommand{\subtitle}[1]{% add subtitle to \maketitle
  \apptocmd{\@title}{\par {\large #1 \par}}{}{}
}
\makeatother
\subtitle{Portfolio Asesmen II-2100 KIPP}
\author{Jonathan Wiguna}
\date{2025-10-18}
\begin{document}
\maketitle

\renewcommand*\contentsname{Table of contents}
{
\hypersetup{linkcolor=}
\setcounter{tocdepth}{2}
\tableofcontents
}

\bookmarksetup{startatroot}

\chapter{Selamat Datang di Portfolio
Saya}\label{selamat-datang-di-portfolio-saya}

Halo! Saya \textbf{Jonathan Wiguna}, mahasiswa Sistem dan Teknologi
Informasi ITB.

Ini adalah portfolio untuk mata kuliah \textbf{II-2100 Komunikasi
Interpersonal dan Publik}.

\section{📚 Navigasi Portfolio}\label{navigasi-portfolio}

Gunakan \textbf{sidebar di sebelah kiri} untuk menjelajahi tugas-tugas
saya:

\subsection{Ujian Tengah Semester
(UTS)}\label{ujian-tengah-semester-uts}

\begin{itemize}
\tightlist
\item
  \textbf{All About Me} - Perkenalan diri saya
\item
  \textbf{My Song for You} - Lagu untuk seseorang yang tidak ada
\item
  \textbf{My Stories for You} - Cerita inspiratif dari perjalanan hidup
\item
  \textbf{My Shapes} - Laporan kepribadian dan pengalaman \emph{(Coming
  Soon)}
\item
  \textbf{My Personal Reviews} - Telaahan pesan personal \emph{(Coming
  Soon)}
\end{itemize}

\subsection{Ujian Akhir Semester (UAS)}\label{ujian-akhir-semester-uas}

\begin{itemize}
\tightlist
\item
  \textbf{My Concepts} - \emph{(Coming Soon)}
\item
  \textbf{My Opinions} - \emph{(Coming Soon)}
\item
  \textbf{My Innovations} - \emph{(Coming Soon)}
\item
  \textbf{My Knowledge} - \emph{(Coming Soon)}
\item
  \textbf{My Professional Reviews} - \emph{(Coming Soon)}
\end{itemize}

\begin{center}\rule{0.5\linewidth}{0.5pt}\end{center}

Terima kasih telah mengunjungi portfolio saya!

\bookmarksetup{startatroot}

\chapter{Tentang Saya}\label{tentang-saya}

\section{Halo, Saya Jonathan}\label{halo-saya-jonathan}

\begin{figure}[H]

{\centering \includegraphics[width=3.125in,height=\textheight,keepaspectratio]{All_About_me/placeholder-photo.jpg}

}

\caption{Foto Profil}

\end{figure}%

\emph{Catatan: Ganti \texttt{placeholder-photo.jpg} dengan foto kamu
sendiri}

\begin{center}\rule{0.5\linewidth}{0.5pt}\end{center}

\section{Siapa Saya?}\label{siapa-saya}

Saya adalah \textbf{Jonathan Wiguna}, mahasiswa semester 7 Sistem dan
Teknologi Informasi di Institut Teknologi Bandung, lahir di Indonesia
tahun 2004. Pada awalnya, saya memilih jurusan ini karena saya ingin
kebebasan: bekerja dari mana saja, kapan saja, dan mengejar impian saya
sendiri.

Namun setelah berpikir lebih jauh, saya menyadari bahwa ada banyak jalan
lain yang juga menawarkan hal yang sama. Akhirnya, selama perkuliahan,
saya menemukan ketertarikan saya pada bidang \textbf{System Analyst}.

\begin{center}\rule{0.5\linewidth}{0.5pt}\end{center}

\section{Apa yang Saya Sukai?}\label{apa-yang-saya-sukai}

Saya menikmati bagaimana sistem bekerja dan bagaimana kebutuhan bisnis
diterjemahkan menjadi solusi teknologi yang efektif. \textbf{UML
(Unified Modeling Language)} adalah salah satu hal yang saya pelajari
dengan antusias, karena saya suka memetakan logika, alur, dan struktur
sebelum sesuatu dibangun.

Di era AI seperti sekarang, saya juga mulai mengeksplorasi bagaimana
teknologi AI dapat diintegrasikan dalam analisis sistem dan automasi
proses bisnis. Masih banyak yang perlu saya pelajari, tapi kombinasi
antara pemikiran analitis dan teknologi baru ini adalah sesuatu yang
membuat saya penasaran untuk terus belajar.

\begin{center}\rule{0.5\linewidth}{0.5pt}\end{center}

\section{Kepribadian Saya}\label{kepribadian-saya}

Saya adalah tipe orang yang \textbf{introverted}, lebih nyaman bekerja
dengan data dan logika daripada berada di keramaian. Meski begitu, saya
menyadari bahwa kemampuan berkomunikasi tetap penting, terutama untuk
menjembatani pemahaman antara berbagai pihak dalam sebuah proyek.

Jujur saja, sebagai mahasiswa semester 7, saya masih sering
bertanya-tanya apakah saya sudah berada di jalur yang benar. Kadang saya
merasa belum cukup ahli, belum cukup percaya diri. Tapi mungkin itulah
bagian dari proses belajar, bukan? \emph{(Self-deprecating humor: karena
kalau tidak diakui, orang lain juga tahu kok)}

\begin{center}\rule{0.5\linewidth}{0.5pt}\end{center}

\section{Penutup}\label{penutup}

Di balik figur saya sebagai mahasiswa IT yang masih mencari jati diri,
ada seseorang yang terus berusaha memahami dunia sistem, teknologi, dan
bagaimana semuanya bisa bekerja lebih baik. Saya mungkin bukan yang
paling menonjol di kelas, tapi saya percaya bahwa setiap orang punya
caranya sendiri untuk berkontribusi.

\begin{center}\rule{0.5\linewidth}{0.5pt}\end{center}

\emph{``The secret of happiness is to see all the marvels of the world,
and never to forget the drops of oil on the spoon.''}\\
--- Paulo Coelho, The Alchemist

\bookmarksetup{startatroot}

\chapter{A Song for Someone Who Doesn't
Exist}\label{a-song-for-someone-who-doesnt-exist}

Sebuah lagu untuk seseorang yang mungkin tidak pernah ada

\hfill\break

\section{Pengantar}\label{pengantar}

Ini adalah lagu untuk seseorang yang belum (atau mungkin tidak akan
pernah) saya temui. Sebuah cerita tentang harapan yang mungkin absurd,
namun tetap terasa nyata di kepala.

\emph{Terinspirasi dari gaya musik TV Girl dan Radiohead---melankolis,
nostalgia, dan sedikit ironis.}

\begin{center}\rule{0.5\linewidth}{0.5pt}\end{center}

\section{\texorpdfstring{\textbf{``Ghost in My
Playlist''}}{``Ghost in My Playlist''}}\label{ghost-in-my-playlist}

\subsection{Verse 1}\label{verse-1}

\begin{verbatim}
I saw you in a coffee shop that doesn't exist  
Sitting by the window with your morning mist  
You looked at me like you knew my name  
But I woke up and you were just a frame  
\end{verbatim}

\subsection{Chorus}\label{chorus}

\begin{verbatim}
You're a ghost in my playlist  
A song I can't forget  
Playing on repeat in my head  
But you're someone I haven't met yet  

You're the one I write about at 3 AM  
When the world is quiet and I'm talking to myself again  
I know it's pathetic, I know it's sad  
But you're the best conversation I never had  
\end{verbatim}

\subsection{Verse 2}\label{verse-2}

\begin{verbatim}
I built you from the movies and the books I read  
A little bit of fiction, a lot of what I need  
You laugh at all my jokes, you understand my silence  
But you live inside my mind—such a perfect alliance  
\end{verbatim}

\subsection{Chorus}\label{chorus-1}

\begin{verbatim}
You're a ghost in my playlist  
A song I can't forget  
Playing on repeat in my head  
But you're someone I haven't met yet  

You're the one I write about at 3 AM  
When the world is quiet and I'm talking to myself again  
I know it's pathetic, I know it's sad  
But you're the best conversation I never had  
\end{verbatim}

\subsection{Bridge}\label{bridge}

\begin{verbatim}
Maybe one day I'll find someone real  
Who makes me forget how this feels  
But until then, I'll keep you here  
In my headphones, crystal clear  
\end{verbatim}

\subsection{Outro}\label{outro}

\begin{verbatim}
So here's to you, whoever you are  
A distant light, a fading star  
I'll keep singing this song in vain  
Until you show up, or I go insane  
\end{verbatim}

\begin{center}\rule{0.5\linewidth}{0.5pt}\end{center}

\section{Refleksi}\label{refleksi}

Ya, saya tahu ini agak menyedihkan. Menulis lagu untuk seseorang yang
tidak ada terdengar seperti plot film indie yang gagal. Tapi kadang
imajinasi lebih aman daripada kenyataan, kan? Setidaknya dalam
imajinasi, tidak ada yang menolak atau mengecewakan.

Mungkin suatu hari nanti, seseorang yang nyata akan muncul dan membuat
lagu ini terasa konyol. Atau mungkin tidak. Either way, ini adalah
bagian dari siapa saya sekarang---seseorang yang masih mencari, masih
berharap, dan masih cukup naif untuk percaya bahwa ada seseorang di luar
sana.

\emph{Dan ya, saya sadar betapa self-deprecating ini kedengarannya. Tapi
setidaknya saya jujur, bukan?}

\begin{center}\rule{0.5\linewidth}{0.5pt}\end{center}

\textbf{Catatan:} Lirik ini dibuat dalam Bahasa Inggris karena lebih
cocok dengan genre musik yang terinspirasi. Jika ingin, bisa
diterjemahkan atau diadaptasi ke Bahasa Indonesia dengan nuansa yang
berbeda.

\bookmarksetup{startatroot}

\chapter{Cerita-Cerita yang Membentuk
Saya}\label{cerita-cerita-yang-membentuk-saya}

Dua momen yang mengajarkan saya tentang kehidupan

\hfill\break

\section{Pengantar}\label{pengantar-1}

Berikut adalah dua cerita dari perjalanan saya yang mungkin terdengar
biasa, tapi sangat berarti bagi saya. Cerita tentang kegagalan,
penerimaan, dan momen-momen ketika saya akhirnya memahami sesuatu yang
sebelumnya tidak pernah saya sadari.

\begin{center}\rule{0.5\linewidth}{0.5pt}\end{center}

\bookmarksetup{startatroot}

\chapter{Cerita 1: Ketika Semua Orang Lolos, Kecuali
Saya}\label{cerita-1-ketika-semua-orang-lolos-kecuali-saya}

\section{Hari Pengumuman SNMPTN}\label{hari-pengumuman-snmptn}

Masih saya ingat dengan jelas hari pengumuman SNMPTN tahun 2022. Saya
duduk di depan laptop, refresh halaman berkali-kali, berharap nama saya
muncul. Tapi tidak ada.

Sementara itu, di grup chat sekolah, notifikasi terus berdatangan:

\begin{quote}
``GUYS AKU LOLOS ITB!''\\
``Alhamdulillah keterima UI!''\\
``SNMPTN berhasil! Thanks buat doa kalian!''
\end{quote}

Saya hanya bisa diam. Mematikan notifikasi. Menatap layar kosong dengan
perasaan yang sulit dijelaskan.

\section{Perasaan Tertinggal}\label{perasaan-tertinggal}

Bukan cuma soal tidak lolos. Yang paling menyakitkan adalah perasaan
\textbf{tertinggal}. Semua teman saya sudah punya kepastian, sementara
saya harus memulai lagi dari nol. Mereka sudah merayakan, sementara saya
harus kembali belajar untuk SBMPTN.

Saya tidak menyalahkan siapa pun. Tapi tetap saja, ada rasa sedih yang
dalam. Kenapa mereka bisa, tapi saya tidak?

\section{Keputusan: Mencoba Lagi}\label{keputusan-mencoba-lagi}

Tapi saya tidak berhenti. Saya punya waktu sekitar \textbf{2 bulan}
untuk mempersiapkan SBMPTN. Jujur, waktu itu sangat singkat dan saya
tidak terlalu yakin. Tapi saya berpikir: \emph{``Apa salahnya mencoba?
Paling tidak, saya sudah berusaha.''}

Saya belajar dengan fokus. Tidak berlebihan, tapi konsisten. Dan yang
paling penting, saya tidak lagi membandingkan diri saya dengan orang
lain. Saya hanya fokus pada apa yang bisa saya lakukan.

\section{Pengumuman SBMPTN: Lolos
ITB}\label{pengumuman-sbmptn-lolos-itb}

Ketika pengumuman SBMPTN keluar, saya hampir tidak percaya.
\textbf{Lolos ITB, Sistem dan Teknologi Informasi.}

Rasanya seperti semua yang saya alami---kekecewaan, kesedihan, kerja
keras---akhirnya membuahkan hasil. Tapi yang lebih penting dari itu,
saya belajar sesuatu yang sangat berharga:

\begin{quote}
\textbf{Tidak ada salahnya mencoba, bahkan ketika peluangnya kecil.}
\end{quote}

Kalau saya menyerah setelah SNMPTN, saya tidak akan ada di ITB sekarang.
Kalau saya terlalu takut gagal lagi, saya tidak akan pernah tahu bahwa
saya sebenarnya bisa.

\section{Pelajaran}\label{pelajaran}

Kegagalan itu menyakitkan. Melihat orang lain berhasil sementara kita
tidak, itu berat. Tapi itulah hidup. Kadang kita harus jatuh dulu
sebelum bisa berdiri lebih kuat.

Dan yang terpenting: \textbf{Jangan pernah berhenti mencoba hanya karena
takut gagal.}

\begin{center}\rule{0.5\linewidth}{0.5pt}\end{center}

\bookmarksetup{startatroot}

\chapter{Cerita 2: Hari Saya Menyadari ITB Itu
Indah}\label{cerita-2-hari-saya-menyadari-itb-itu-indah}

\section{Setelah Kuis Fisika yang
Menghancurkan}\label{setelah-kuis-fisika-yang-menghancurkan}

Semester 2, saya baru saja selesai mengerjakan kuis Fisika Dasar yang
sangat sulit. Saya keluar dari kelas dengan perasaan hancur. Saya tahu
saya tidak mengerjakan dengan baik. Pikiran saya dipenuhi dengan
kekhawatiran: \emph{``Apakah saya cukup pintar untuk bertahan di
sini?''}

Saya berjalan keluar gedung, masih terbayang-bayang soal yang tidak bisa
saya jawab.

\section{Momen di Jalan Pulang}\label{momen-di-jalan-pulang}

Tapi kemudian, saya berhenti sejenak. Saya melihat sekeliling.

Langit sore itu biru dengan awan yang bergerak perlahan. Pohon-pohon di
sekitar kampus bergoyang tertiup angin. Mahasiswa lain berjalan santai,
tertawa, duduk di taman, atau sekadar ngobrol.

Dan tiba-tiba saya menyadari sesuatu:

\begin{quote}
\textbf{ITB itu indah.}
\end{quote}

Bukan hanya gedungnya, bukan hanya prestisenya. Tapi \textbf{momen-momen
kecil} seperti ini. Langit sore. Angin sepoi-sepoi. Kehidupan yang terus
berjalan meskipun saya baru saja gagal dalam satu kuis.

\section{Quote dari The Alchemist}\label{quote-dari-the-alchemist}

Saat itu saya teringat sebuah cerita dari buku \textbf{The Alchemist}
karangan Paulo Coelho. Ada cerita tentang seorang anak muda yang diberi
sendok berisi minyak dan diminta berjalan keliling istana tanpa
menumpahkan minyaknya.

Pertama kali, dia terlalu fokus pada sendok sehingga tidak melihat
keindahan istana. Kedua kalinya, dia terlalu fokus pada istana sehingga
minyaknya tumpah.

Pelajarannya:

\begin{quote}
``The secret of happiness is to see all the marvels of the world, and
never to forget the drops of oil on the spoon.''
\end{quote}

Artinya, kebahagiaan ada pada \textbf{keseimbangan}: menikmati keindahan
dunia, tapi tidak melupakan tanggung jawab kita.

\section{Pelajaran yang Saya Dapat}\label{pelajaran-yang-saya-dapat}

Selama dua semester, saya terlalu fokus pada ``sendok''---nilai, tugas,
ujian. Saya lupa untuk melihat ``istana''---keindahan kampus,
teman-teman, pengalaman belajar itu sendiri.

Hari itu, setelah kuis Fisika yang buruk, saya akhirnya menyadarinya.
ITB bukan hanya tentang akademik. ITB adalah tentang \textbf{proses}.
Tentang belajar, jatuh, bangkit, dan menikmati perjalanan.

\section{Refleksi}\label{refleksi-1}

Sekarang, setiap kali saya merasa overwhelmed dengan tugas atau nilai,
saya mencoba untuk berhenti sejenak. Melihat sekeliling. Mengingat bahwa
hidup bukan hanya tentang ``drops of oil'' yang harus dijaga, tapi juga
tentang ``marvels of the world'' yang harus dinikmati.

\begin{center}\rule{0.5\linewidth}{0.5pt}\end{center}

\section{Penutup}\label{penutup-1}

Dua cerita ini mungkin terdengar sederhana. Tapi bagi saya, ini adalah
momen-momen yang membentuk siapa saya sekarang. Saya belajar untuk tidak
menyerah. Saya belajar untuk melihat keindahan di tengah kesulitan. Dan
saya belajar bahwa kehidupan adalah tentang keseimbangan.

\emph{Dan ya, saya masih sering gagal menjaga keseimbangan itu. Tapi
setidaknya sekarang saya tahu bahwa itu penting.}

\bookmarksetup{startatroot}

\chapter{UTS-4 My SHAPE (Spiritual Gifts, Heart, Abilities, Personality,
Experiences)}\label{uts-4-my-shape-spiritual-gifts-heart-abilities-personality-experiences}

\begin{quote}
\textbf{Tujuan:} Merangkum rancangan diri (charter) agar saya melayani,
berkarya, dan memimpin secara paling selaras dengan karunia dan
pengalaman hidup saya. Dapat langsung ditempel ke halaman \textbf{UTS-4
--- My SHAPE} dan dipakai sebagai acuan aksi 90 hari.
\end{quote}

\section{\texorpdfstring{Sumber
\href{StrengthsProfile-Armein-Langi.pdf}{VIA
assessment}}{Sumber VIA assessment}}\label{sumber-via-assessment}

\section{0) Ringkasan 1 Halaman}\label{ringkasan-1-halaman}

\textbf{Peran Inti:} Profesor \& Elder --- perancang ekosistem
belajar-bernilai, pembimbing, dan pemimpin pelayanan komunitas.
\textbf{Misi:} Mengangkat kualitas hidup melalui \emph{smart
engineering} dan \emph{value-oriented education}, khususnya bagi
lansia/keluarga/komunitas (GRACE), serta pelayanan gerejawi yang
menumbuhkan kasih dan pengharapan. \textbf{Kekuatan Utama:} mengkonsep
sistem utuh, menulis \& mengajar, membangun jejaring, merancang
rubrik/alat evaluasi, menggerakkan proyek lintas-disiplin.
\textbf{Dampak yang Dituju:} karya, kurikulum, dan pelayanan yang
menumbuhkan karakter, keterampilan, serta kesejahteraan berkeadilan.

\textbf{Peta SHAPE (singkat):}

\begin{itemize}
\tightlist
\item
  \textbf{S --- Spiritual Gifts:} Teaching, Shepherding/Pastoring,
  Leadership, Wisdom/Discernment, Exhortation/Encouragement,
  Administration.
\item
  \textbf{H --- Heart (Minat \& Cinta Pelayanan):} pendidikan
  berorientasi nilai; kesejahteraan lansia \& keluarga (GRACE);
  pembinaan iman; menulis kisah/novel/lirik; rekayasa cerdas \& AI untuk
  kebaikan bersama; mentorship mahasiswa-dosen; penguatan jemaat.
\item
  \textbf{A --- Abilities (Kemampuan):} perancangan sistem (PSKVE/TISE),
  kurikulum \& rubrik, riset \& publikasi, menulis multi-format
  (Quarto/LaTeX), pemrograman (Python/R/Prolog/Modelica), komunikasi
  publik, memimpin kolaborasi.
\item
  \textbf{P --- Personality (Gaya Kepribadian Kerja):} strategis \&
  reflektif, berorientasi visi \& nilai, analitis-sistemik, kolaboratif,
  tenang dalam krisis, suka membangun standar \& alat.
\item
  \textbf{E --- Experiences (Pengalaman Kunci):} dosen \& peneliti
  lintas proyek (GRACE, Smart Engineering, pendidikan), Elder \&
  pengorganisasi jemaat, penulis kreatif, arsitek sistem pengetahuan
  (Obsidian/GitHub/Quarto), penggerak sarasehan \& penggalangan
  dukungan.
\end{itemize}

\begin{center}\rule{0.5\linewidth}{0.5pt}\end{center}

\section{1) S --- Spiritual Gifts (Karunia
Rohani)}\label{s-spiritual-gifts-karunia-rohani}

\begin{itemize}
\tightlist
\item
  \textbf{Teaching \& Wisdom/Discernment:} mengubah konsep kompleks
  menjadi peta belajar, rubrik, dan alat evaluasi yang memampukan.
\item
  \textbf{Shepherding/Pastoring \& Exhortation:} membimbing
  individu/kelompok dengan empati, meneguhkan, dan memberi arah.
\item
  \textbf{Leadership \& Administration:} merancang ekosistem
  (orang--proses--alat) dengan target berdampak dan terukur.
\end{itemize}

\textbf{Indikator Bukti:} silabus \& rubrik (II-2100/EL2007), naskah
pengajaran, bimbingan riset, modul/website kelas, program jemaat.

\begin{center}\rule{0.5\linewidth}{0.5pt}\end{center}

\section{2) H --- Heart (Minat, Nilai,
Kepedulian)}\label{h-heart-minat-nilai-kepedulian}

\begin{itemize}
\tightlist
\item
  Pendidikan yang \textbf{mencipta nilai} (CPMK↔rubrik↔artefak nyata).
\item
  \textbf{GRACE}: kualitas hidup lansia/keluarga melalui sistem dukung
  cerdas \& komunitas saling-melayani.
\item
  \textbf{Gereja \& Komunitas}: penguatan iman, kesalingan, dan
  pelayanan kasih.
\item
  \textbf{Kreativitas naratif}: kisah/novel/lirik sebagai sarana edukasi
  \& pengharapan.
\item
  \textbf{Rekayasa cerdas \& AI} untuk kemaslahatan.
\end{itemize}

\textbf{Masalah yang ingin dipecahkan:} kesenjangan antara
pengetahuan--karakter--aksi; pembelajaran kurang bermakna; layanan
komunitas belum terukur dampaknya.

\begin{center}\rule{0.5\linewidth}{0.5pt}\end{center}

\section{3) A --- Abilities (Kemampuan
Andal)}\label{a-abilities-kemampuan-andal}

\begin{itemize}
\tightlist
\item
  \textbf{Perancangan sistem} (PSKVE/TISE), \emph{value co‑creation},
  finansial rekayasa, desain instrumen penilaian.
\item
  \textbf{Kurikulum \& pedagogi}: CPMK↔rubrik↔tugas↔bukti; otomasi alur
  kerja (Python/Quarto/GitHub).
\item
  \textbf{Riset \& penulisan ilmiah}; \textbf{karya kreatif} (prosa,
  lirik, ceramah/khotbah).
\item
  \textbf{Teknis}: Python, R, Prolog (ontologi), Modelica, Quarto/LaTeX,
  Obsidian, GitHub, Graphviz.
\item
  \textbf{Komunikasi \& kepemimpinan}: orasi publik, fasilitasi
  sarasehan, negosiasi kolaborasi.
\end{itemize}

\begin{center}\rule{0.5\linewidth}{0.5pt}\end{center}

\section{4) P --- Personality (Gaya Kerja \&
Kolaborasi)}\label{p-personality-gaya-kerja-kolaborasi}

\begin{itemize}
\tightlist
\item
  \textbf{Strategis‑sistemik} (melihat gambaran besar, memetakan
  bagian-bagian).
\item
  \textbf{Reflektif \& nilai‑driven} (standar etis \& mutu).
\item
  \textbf{Kolaboratif} (membangun jejaring, memberi ruang tumbuh).
\item
  \textbf{Tenang‑tangguh} (fokus hasil jangka panjang).
\item
  \textbf{Pembelajar \& pembuat alat} (suka membuat template, rubrik,
  pipeline).
\end{itemize}

\begin{center}\rule{0.5\linewidth}{0.5pt}\end{center}

\section{5) E --- Experiences (Pengalaman
Pembentuk)}\label{e-experiences-pengalaman-pembentuk}

\begin{itemize}
\tightlist
\item
  \textbf{Akademik \& Riset:} merancang mata kuliah, SLR AI \&
  transformasi digital, proyek GRACE \& Smart Engineering.
\item
  \textbf{Pelayanan \& Organisasi:} Elder GKI, fasilitator sarasehan,
  penggalangan dukungan jemaat, pembinaan rohani.
\item
  \textbf{Kreasi Konten:} penulisan novel/khotbah/lirik; produksi materi
  ajar multi‑format.
\item
  \textbf{Infrastruktur Pengetahuan:} Obsidian--GitHub--Quarto, rubrik
  otomatis, bank soal.
\end{itemize}

\textbf{Pelajaran Inti:} integrasi iman--ilmu--nilai; sistem yang baik
melipatgandakan orang baik; narasi menggerakkan aksi.

\begin{center}\rule{0.5\linewidth}{0.5pt}\end{center}

\section{6) Piagam Diri (Self‑Charter)}\label{piagam-diri-selfcharter}

\textbf{Misi Hidup:} merancang dan menggerakkan ekosistem pembelajaran
\& pelayanan yang memerdekakan, bermakna, dan berkeadilan. \textbf{Nilai
Inti:} kasih, integritas, kebijaksanaan, keberanian, mutu, keberpihakan
pada yang lemah. \textbf{Peran Inti:} Perancang sistem
nilai‑pembelajaran; Pembimbing \& pengajar; Pemimpin pelayanan
komunitas. \textbf{Kompas Keputusan:} (1) Dampak pada manusia; (2)
Keselarasan nilai; (3) Keberlanjutan; (4) Kemampuan tim mengelola; (5)
Bukti terukur. \textbf{Janji Pelayanan:} hadir dengan empati, mendengar,
memberi arah praktis, membangun alat agar orang lain bertumbuh.
\textbf{Batasan:} menolak proyek yang mengabaikan martabat
manusia/etika; menjaga ritme kerja‑istirahat‑keluarga.

\begin{center}\rule{0.5\linewidth}{0.5pt}\end{center}

\section{7) Narasi 90 Detik (Elevator
Pitch)}\label{narasi-90-detik-elevator-pitch}

``Saya seorang profesor dan elder yang merancang ekosistem belajar dan
pelayanan berbasis nilai. Karunia saya mengajar, membimbing, dan
memimpin dengan pendekatan sistem: mengubah konsep besar menjadi peta,
rubrik, dan alat yang membuat orang bertumbuh. Hati saya pada pendidikan
bermakna, kesejahteraan lansia dan keluarga, serta penguatan jemaat.
Dengan pengalaman lintas riset, kurikulum, dan pelayanan, saya
menghubungkan ilmu, iman, dan aksi. Target saya sederhana: menghadirkan
karya dan komunitas yang saling menguatkan---di kelas, di gereja, dan di
masyarakat---agar lebih banyak orang hidup berkualitas, berpengharapan,
dan siap melayani.''

\begin{center}\rule{0.5\linewidth}{0.5pt}\end{center}

\section{8) Service‑Fit Map (Tempat Saya Paling
Berdampak)}\label{servicefit-map-tempat-saya-paling-berdampak}

\begin{itemize}
\tightlist
\item
  \textbf{Kampus:} perancangan kurikulum \& rubrik; mentorship riset;
  otomasi pipeline belajar; kuliah \& orasi.
\item
  \textbf{Jemaat:} pembinaan rohani \& khotbah; fasilitasi sarasehan;
  program lansia/keluarga (GRACE).
\item
  \textbf{Riset‑Inovasi:} desain platform nilai‑ciptakan (PSKVE);
  publikasi; konsorsium kolaborasi.
\item
  \textbf{Kreasi Naratif:} kisah/lirik sebagai media edukasi \&
  penguatan batin.
\end{itemize}

\begin{center}\rule{0.5\linewidth}{0.5pt}\end{center}

\section{9) Evidences (Artefak \&
Tautan)}\label{evidences-artefak-tautan}

\begin{quote}
Ganti tanda {[} {]} dengan tautan/berkas Anda.
\end{quote}

\begin{itemize}
\tightlist
\item[$\square$]
  Silabus \& rubrik II‑2100 / EL2007
\item[$\square$]
  Modul/website kelas \& bank soal
\item[$\square$]
  Khotbah/renungan \& materi sarasehan
\item[$\square$]
  Publikasi/SLR \& proposal riset (GRACE, dsb.)
\item[$\square$]
  Novel/lirik \& materi kreatif
\item[$\square$]
  Pipeline otomasi (Quarto/GitHub/Obsidian)
\end{itemize}

\begin{center}\rule{0.5\linewidth}{0.5pt}\end{center}

\section{10) Rencana Aksi 90 Hari
(SMART)}\label{rencana-aksi-90-hari-smart}

\begin{enumerate}
\def\labelenumi{\arabic{enumi}.}
\tightlist
\item
  \textbf{Rampungkan halaman UTS (KIPP/All‑About‑Me) end‑to‑end.}
  \emph{Outcome:} semua tugas berisi bukti + rubrik; \emph{Due:} T‑14
  hari.
\item
  \textbf{Mentor 3 tim mahasiswa menyusun artefak bernilai.}
  \emph{Outcome:} 3 proyek dengan metrik dampak; \emph{Due:} T‑45 hari.
\item
  \textbf{Pilot GRACE micro‑service di jemaat.} \emph{Outcome:} 1
  layanan kecil terukur (mis. pendampingan lansia); \emph{Due:} T‑90
  hari.
\item
  \textbf{Publikasi ringkas (working paper) integrasi
  iman--ilmu--nilai.} \emph{Outcome:} 1 naskah pra‑cetak; \emph{Due:}
  T‑75 hari.
\end{enumerate}

\begin{center}\rule{0.5\linewidth}{0.5pt}\end{center}

\section{11) SHAPE ↔ CPMK (Interpersonal \& Public
Communication)}\label{shape-cpmk-interpersonal-public-communication}

\begin{itemize}
\tightlist
\item
  \textbf{Self‑awareness \& refleksi (CPMK‑S):} dituangkan pada Piagam
  Diri \& Narasi 90 detik.
\item
  \textbf{Empati \& komunikasi etis (CPMK‑E):} Shepherding/Exhortation →
  khotbah, mentoring, review berempati.
\item
  \textbf{Storytelling \& presentasi (CPMK‑P):} Teaching + kreasi
  naratif → kuliah, cerita, lirik.
\item
  \textbf{Kolaborasi \& kepemimpinan (CPMK‑K):}
  Leadership/Administration → proyek riset/komunitas terukur.
\end{itemize}

\begin{center}\rule{0.5\linewidth}{0.5pt}\end{center}

\section{12) Self‑Assessment Rubrik UTS‑4 (isi
skormu)}\label{selfassessment-rubrik-uts4-isi-skormu}

\begin{longtable}[]{@{}
  >{\raggedright\arraybackslash}p{(\linewidth - 6\tabcolsep) * \real{0.3382}}
  >{\raggedright\arraybackslash}p{(\linewidth - 6\tabcolsep) * \real{0.4412}}
  >{\raggedleft\arraybackslash}p{(\linewidth - 6\tabcolsep) * \real{0.1471}}
  >{\raggedright\arraybackslash}p{(\linewidth - 6\tabcolsep) * \real{0.0735}}@{}}
\toprule\noalign{}
\begin{minipage}[b]{\linewidth}\raggedright
Kriteria
\end{minipage} & \begin{minipage}[b]{\linewidth}\raggedright
Deskripsi
\end{minipage} & \begin{minipage}[b]{\linewidth}\raggedleft
Skor (1--5)
\end{minipage} & \begin{minipage}[b]{\linewidth}\raggedright
Bukti
\end{minipage} \\
\midrule\noalign{}
\endhead
\bottomrule\noalign{}
\endlastfoot
Kelengkapan SHAPE & S‑H‑A‑P‑E jelas \& terisi & & \\
Koherensi Piagam Diri & misi‑nilai‑peran konsisten & & \\
Narasi 90 detik & ringkas, kuat, mengundang aksi & & \\
Evidence \& Aksi 90 hari & tautan bukti \& rencana SMART & & \\
\end{longtable}

\textbf{Total (maks 20):} {[} {]} \textbf{Tingkat:} ☐ A (≥85\%) ☐ B
(70--84\%) ☐ C (60--69\%) ☐ D (50--59\%) ☐ E (\textless50\%)

\begin{center}\rule{0.5\linewidth}{0.5pt}\end{center}

\section{13) Versi Ultra‑Ringkas (≤140
kata)}\label{versi-ultraringkas-140-kata}

``Saya profesor \& elder dengan karunia mengajar, membimbing, dan
memimpin secara sistemik. Hati saya pada pendidikan bernilai,
kesejahteraan lansia/keluarga (GRACE), dan penguatan jemaat. Kemampuan
saya merancang kurikulum, rubrik, dan alat otomasi belajar; menulis
ilmiah \& kreatif; serta menggerakkan kolaborasi. Pengalaman saya di
kampus, gereja, riset, dan kreasi konten mengajarkan integrasi
iman--ilmu--aksi. Misi saya menghadirkan ekosistem yang memerdekakan: di
kelas melalui pembelajaran bermakna; di jemaat melalui pelayanan kasih
yang terukur; dan di masyarakat melalui inovasi yang adil. Target 90
hari: menuntaskan artefak UTS, mementori 3 tim mahasiswa, memulai
layanan mikro GRACE, dan menerbitkan naskah ringkas.''

\section{Piagam Diri --- Armein Z. R.
Langi}\label{piagam-diri-armein-z.-r.-langi}

\textbf{Pernyataan Misi} Saya adalah insinyur-pendidik dan penulis yang
menyalakan sukacita belajar, menumbuhkan empati, dan merancang sistem
cerdas yang memuliakan Tuhan serta meningkatkan kualitas hidup keluarga,
kampus, dan komunitas. (Struktur mengikuti kerangka \emph{My
SHAPE}---Piagam Diri 1-halaman. )

\textbf{S --- Signature Strengths (inti kekuatan khas)} Humor,
Spiritualitas, Kreativitas, Suka Belajar, Keingintahuan, Pandangan
(wisdom/perspective), Bersyukur, Keadilan, Kecerdasan Sosial, Kejujuran,
Kepemimpinan. (Sumber: VIA Character Strengths Profile, 13 Okt 2025. )

\textbf{H --- Heart (nilai \& panggilan)} Empati sebagai kecerdasan
tertinggi; kebaikan lebih utama daripada sekadar pintar; pencarian ``The
True Reality''; sukacita hidup yang mengasihi; keluarga sebagai
ekosistem kasih. (Disimpulkan dari tulisan-tulisan Anda di blog:
\emph{Empati: Kecerdasan Tertinggi}; \emph{On Being Nice}; \emph{The
Truth, The True Reality}; tagline blog; catatan keluarga.
(\href{https://ii-2100.github.io/all-about-me/}{Armein Z. R. Langi in
the City of Eden}))

\textbf{A --- Aptitudes \& Acquired Skills (bakat \& keterampilan
kunci)} Perancangan \& penelitian sistem/komputasi (speech compression,
FPGA), rekayasa \& kurikulum, kepemimpinan akademik, penulisan \&
penceritaan, fasilitasi pembelajaran, sistem \& organisasi. (Contoh
teknis: riset speech compression \& desain kontrol prosesor pada awal
karier.
(\href{https://ii-2100.github.io/all-about-me/My_Song_for_You/index.html}{Armein
Z. R. Langi in the City of Eden}))

\textbf{P --- Personality (gaya kerja yang menonjol)}
Reflektif-analitis, empatik-inklusif, visioner, pembelajar antusias,
kolaboratif; berpihak pada keadilan \& integritas. (Disintesis dari pola
kekuatan VIA dan tema tulisan Anda. )

\textbf{E --- Experiences (jejak pembentuk identitas)}

\begin{itemize}
\tightlist
\item
  \textbf{Ketangguhan pribadi} --- ``The Child Who Learned to Walk at
  the Disneyland'': ketekunan, berjalan dalam dingin, terus melangkah
  menuju tujuan.
  (\href{https://ii-2100.github.io/all-about-me/My_Stories_for_You/index.html}{Armein
  Z. R. Langi in the City of Eden})
\item
  \textbf{Lompatan kompetensi awal} --- perjalanan riset: software
  speech compression jadi dalam 3 bulan; desain chip kontrol di Xilinx
  FPGA; menulis paper.
  (\href{https://ii-2100.github.io/all-about-me/My_Song_for_You/index.html}{Armein
  Z. R. Langi in the City of Eden})
\item
  \textbf{Keluarga \& komunitas} --- keluarga besar sebagai sumber
  nilai, pelayanan, dan sukacita.
  (\href{https://ii-2100.github.io/all-about-me/My_Shapes/index.html}{Armein
  Z. R. Langi in the City of Eden})
\item
  \textbf{Standar keunggulan} --- sensibilitas benchmarking sains \&
  pendidikan (refleksi tentang Caltech).
  (\href{https://ii-2100.github.io/all-about-me/My_Personal_Reviews/index.html}{Armein
  Z. R. Langi in the City of Eden})
\end{itemize}

\textbf{Janji Praktis (Operating Principles)}

\begin{enumerate}
\def\labelenumi{\arabic{enumi}.}
\tightlist
\item
  \emph{People first with empathy} • 2) \emph{Truth-seeking with
  humility} • 3) \emph{Design for value \& justice} • 4) \emph{Teach
  what I practice, practice what I teach} • 5) \emph{Joyful learning,
  faithful living}. (Kerangka dan cara merangkum diadaptasi dari
  \emph{My SHAPE Toolkit}. )
\end{enumerate}

\begin{center}\rule{0.5\linewidth}{0.5pt}\end{center}

\section{Narasi Diri (versi 90 detik)}\label{narasi-diri-versi-90-detik}

Saya Armein---insinyur, pendidik, dan penulis---yang percaya bahwa
pengetahuan hanya bermakna bila melahirkan kasih dan keadilan. Kekuatan
saya adalah \textbf{spiritualitas yang membumi, kreativitas rekayasa,
dan kegembiraan belajar tanpa henti}, yang saya pakai untuk menyalakan
semangat orang lain.

Perjalanan saya ditempa oleh pengalaman yang mengajarkan
\textbf{ketekunan}---mulai dari ``berjalan dalam dingin'' hingga tuntas
menyelesaikan riset komputasi dan merancang sistem sejak awal karier.
Keluarga dan komunitas menjadi ekosistem kasih tempat saya belajar bahwa
\textbf{kebaikan lebih tinggi nilainya daripada sekadar pintar} dan
\textbf{empati adalah kecerdasan tertinggi}.
(\href{https://ii-2100.github.io/all-about-me/My_Stories_for_You/index.html}{Armein
Z. R. Langi in the City of Eden})

Ke depan, saya ingin terus \textbf{mendesain lingkungan belajar dan
sistem cerdas} yang memuliakan Tuhan dan membawa berkat
nyata---membentuk insan pembelajar yang jujur, adil, dan penuh
syukur---seraya menjaga sukacita: \emph{joy of loving and exciting
life}.
(\href{https://ii-2100.github.io/all-about-me/My_Stories_for_You/index.html}{Armein
Z. R. Langi in the City of Eden})

\begin{center}\rule{0.5\linewidth}{0.5pt}\end{center}

\section{Narasi Diri (versi panjang, 3--5
paragraf)}\label{narasi-diri-versi-panjang-35-paragraf}

\textbf{Kini.} Saya mengabdikan diri sebagai insinyur-pendidik yang
merancang pengalaman belajar dan sistem cerdas agar manusia bertumbuh
utuh: cakap teknis, peka nurani, dan gembira belajar. Kekuatan
saya---spiritualitas, kreativitas, suka belajar, keingintahuan,
perspektif, keadilan, dan kepemimpinan---mengarahkan cara saya memimpin,
mengajar, dan menulis.

\textbf{Dulu---titik balik.} Saya belajar bahwa langkah kecil yang
konsisten mengalahkan rintangan besar: berjalan sendirian dalam
dingin---secara harfiah dan metaforis---membentuk ketahanan batin. Di
laboratorium, saya menuntaskan perangkat lunak \textbf{speech
compression} dalam waktu singkat dan merancang \textbf{control unit}
berbasis FPGA, lalu menulis paper pertama---momen yang mengajarkan
disiplin, standar mutu, dan keberanian intelektual.
(\href{https://ii-2100.github.io/all-about-me/My_Stories_for_You/index.html}{Armein
Z. R. Langi in the City of Eden})

\textbf{Nilai yang saya pegang.} Saya memilih \textbf{kebaikan} di atas
sekadar \textbf{kepintaran}, menempatkan \textbf{empati} sebagai
kecerdasan tertinggi, dan mengejar \textbf{kebenaran sebagai realitas
yang sesungguhnya}. Keluarga besar meneguhkan panggilan itu---rumah
sebagai laboratorium kasih dan keadilan.
(\href{https://azrl.wordpress.com/2010/01/09/on-being-nice/?utm_source=chatgpt.com}{Armein
Z. R. Langi in the City of Eden})

\textbf{Ke depan.} Dengan kerangka \textbf{My SHAPE}, saya terus
menyelaraskan kekuatan, nilai, keterampilan, gaya kerja, dan pengalaman
untuk merancang program, riset, dan karya tulis yang berdampak---agar
teknologi menjadi pelayan kemanusiaan, dan setiap ruang belajar menjadi
\emph{City of Eden}: tempat sukacita, kasih, dan keunggulan bertemu.

\bookmarksetup{startatroot}

\chapter{UTS-5 My Personal Reviews}\label{uts-5-my-personal-reviews}

Berikut cara saya melakukan review: mengguan chatGPT, saya mengattach
\href{skor_uts.pdf}{file promt ChatGPT}, disertai perintah :``self
assess uts-1 sanpai uts-5 dari URL
`https://ii-2100.github.io/all-about-me/'\,''

ChatGPT melakukan self-assessment UTS-1 s.d. UTS-5 langsung dari laman
yang Anda berikan dan menilai memakai rubrik tugas UTS (skala 1--5 per
kriteria). Rekap skor siap diunduh sebagai CSV:
\href{sandbox:/mnt/data/UTS_self_assessment.csv}{Download CSV
ringkasan}.

\bookmarksetup{startatroot}

\chapter{Hasil Self-Assessment UTS (URL:
ii-2100.github.io/all-about-me)}\label{hasil-self-assessment-uts-url-ii-2100.github.ioall-about-me}

\section{Identifikasi}\label{identifikasi}

\begin{itemize}
\tightlist
\item
  Nama \& NIM penulis: \textbf{Armein Z. R. Langi -- 131902360} (tertera
  di halaman depan portofolio).
  (\href{https://ii-2100.github.io/all-about-me/}{II 2100})
\item
  Penilai: \textbf{Self-assessment (Armein Z. R. Langi)}
\item
  Catatan cakupan: halaman beranda memuat ``About Me''; navigasi ke ``My
  Songs for You'', ``My Stories for You'', ``My Shapes'', dan ``My
  Personal Reviews'' tersedia.
  (\href{https://ii-2100.github.io/all-about-me/}{II 2100})
\end{itemize}

\section{Tinjauan Umum}\label{tinjauan-umum}

\begin{itemize}
\tightlist
\item
  \textbf{UTS-1 (All About Me)} hadir di beranda (``Selamat Berjumpa /
  About Me''). Isi memperkenalkan identitas dan latar personal secara
  padat. (\href{https://ii-2100.github.io/all-about-me/}{II 2100})
\item
  \textbf{UTS-2 (My Songs for You)} memuat judul karya dan tautan audio,
  namun lirik/isi tidak ditampilkan di halaman (file audio tidak bisa
  saya akses dari sini), sehingga penilaian konten terbatas pada
  kelengkapan presentasi.
  (\href{https://ii-2100.github.io/all-about-me/My_Song_for_You/index.html}{II
  2100})
\item
  \textbf{UTS-3 (My Stories for You)} berisi tautan ke beberapa cerita;
  konten naratif kuat, terutama pada kisah keluarga ``Gaun Pengantin
  Gladys'' serta kisah inspiratif lainnya.
  (\href{https://ii-2100.github.io/all-about-me/My_Stories_for_You/index.html}{II
  2100})
\item
  \textbf{UTS-4 (My SHAPE)} saat ini halaman kosong/tanpa isi, hanya
  navigasi.
  (\href{https://ii-2100.github.io/all-about-me/My_Shapes/index.html}{II
  2100})
\item
  \textbf{UTS-5 (My Personal Reviews)} berisi metode/tautan panduan
  review, namun belum ada contoh \textbf{review personal} yang lengkap
  terhadap sebuah pesan/teks.
  (\href{https://ii-2100.github.io/all-about-me/My_Personal_Reviews/index.html}{II
  2100})
\end{itemize}

\begin{center}\rule{0.5\linewidth}{0.5pt}\end{center}

\section{Tinjauan Spesifik + Skor
(1--5)}\label{tinjauan-spesifik-skor-15}

\subsection{UTS-1 --- All About Me (di
beranda)}\label{uts-1-all-about-me-di-beranda}

\textbf{Skor per kriteria:} Orisinalitas \textbf{3}, Keterlibatan
\textbf{3}, Humor \textbf{1}, Wawasan/Insight \textbf{3} → \textbf{Total
10/20 (50\%)}. \textbf{Alasan singkat:} Perkenalan diri faktual dan
jelas, namun gaya masih informatif ketimbang naratif (minim humor dan
moment ``hook''). (\href{https://ii-2100.github.io/all-about-me/}{II
2100}) \textbf{Saran perbaikan:} Buka dengan anekdot pribadi (1--2
paragraf) yang ``mengikat'' (mis. titik balik karier/keluarga),
tambahkan satu momen humor ringan, lalu tutup dengan refleksi singkat
tentang nilai/visi diri agar aspek insight naik.

\subsection{UTS-2 --- My Songs for You}\label{uts-2-my-songs-for-you}

\textbf{Skor per kriteria:} Orisinalitas \textbf{2}, Keterlibatan
\textbf{2}, Humor \textbf{1}, Inspirasi \textbf{2} → \textbf{Total 7/20
(35\%)}. \textbf{Alasan singkat:} Halaman menampilkan judul lagu ``River
in my Mind'', ``Heaven on Earth'', namun tanpa lirik/cerita di balik
lagu sehingga sulit menilai aspek pesan, humor, dan inspirasi.
(\href{https://ii-2100.github.io/all-about-me/My_Song_for_You/index.html}{II
2100}) \textbf{Saran perbaikan:} Tambahkan lirik lengkap, 1 paragraf
cerita proses kreatif, dan 2--3 kalimat ``pesan untukmu'' agar inspirasi
terbaca; sertakan player/tautan yang dapat diputar langsung + fallback
transkrip.

\subsection{UTS-3 --- My Stories for
You}\label{uts-3-my-stories-for-you}

\textbf{Skor per kriteria:} Orisinalitas \textbf{5}, Keterlibatan
\textbf{5}, Pengembangan Narasi \textbf{4}, Inspirasi \textbf{5} →
\textbf{Total 19/20 (95\%)}. \textbf{Alasan singkat:} Cerita ``Gaun
Pengantin Gladys'' dkk sangat personal, emosional, dan inspiratif; ritme
bertutur hidup serta detail situasional kuat (konten ramu unsur
penebusan/keluarga/iman).
(\href{https://ii-2100.github.io/all-about-me/My_Stories_for_You/index.html}{II
2100}) \textbf{Saran perbaikan:} Tambah ``lead'' 2--3 kalimat yang
merangkum pesan kunci tiap cerita; akhiri dengan ajakan/refleksi 1--2
kalimat agar resonansi ke pembaca makin jelas.

\subsection{UTS-4 --- My SHAPE}\label{uts-4-my-shape}

\textbf{Skor per kriteria:} Orisinalitas \textbf{1}, Keterlibatan
\textbf{1}, Pengembangan \textbf{1}, Inspirasi \textbf{1} →
\textbf{Total 4/20 (20\%)}. \textbf{Alasan singkat:} Konten belum
tersedia.
(\href{https://ii-2100.github.io/all-about-me/My_Shapes/index.html}{II
2100}) \textbf{Saran perbaikan (prioritas):}

\begin{enumerate}
\def\labelenumi{\arabic{enumi}.}
\tightlist
\item
  Tampilkan rekap SHAPE (Spiritual gifts, Heart, Abilities, Personality,
  Experiences) dalam 1 tabel ringkas + 3--5 butir bukti;
\item
  Narasikan 2 pengalaman kunci (±150 kata/each) yang menunjukkan
  ``signature strengths'';
\item
  Tutup dengan 3 rencana aksi konkret (who/what/when).
\end{enumerate}

\subsection{UTS-5 --- My Personal
Reviews}\label{uts-5-my-personal-reviews-1}

\textbf{Skor per kriteria:} Pemahaman Konsep \textbf{2}, Analisis Kritis
\textbf{1}, Argumentasi (Logos) \textbf{1}, Etos \& Empati \textbf{2},
Rekomendasi \textbf{1} → \textbf{Total 7/25 (28\%)}. \textbf{Alasan
singkat:} Halaman berisi metode cara menilai, tetapi \textbf{belum ada}
contoh \textbf{review personal} yang lengkap terhadap sebuah pesan/teks
sehingga aspek analisis-argumentasi tak terbaca.
(\href{https://ii-2100.github.io/all-about-me/My_Personal_Reviews/index.html}{II
2100}) \textbf{Saran perbaikan:} Pilih 1 karya personal (mis.
UTS-1/2/3), tulis review 400--600 kata: ringkas pesan, nilai dengan
rubrik, berikan 2--3 bukti kutipan, evaluasi etos/empati, lalu tutup
dengan rekomendasi perbaikan spesifik.

\begin{center}\rule{0.5\linewidth}{0.5pt}\end{center}

\section{Rekap Skor (ringkas)}\label{rekap-skor-ringkas}

\begin{itemize}
\tightlist
\item
  \textbf{UTS-1:} 10/20 → \textbf{50\%}
\item
  \textbf{UTS-2:} 7/20 → \textbf{35\%}
\item
  \textbf{UTS-3:} 19/20 → \textbf{95\%}
\item
  \textbf{UTS-4:} 4/20 → \textbf{20\%}
\item
  \textbf{UTS-5:} 7/25 → \textbf{28\%}
\end{itemize}

CSV lengkap sudah saya siapkan untuk dokumentasi dan olah lanjut:
\href{sandbox:/mnt/data/UTS_self_assessment.csv}{Download CSV
ringkasan}.

\section{Langkah Perbaikan Cepat (prioritas 1
minggu)}\label{langkah-perbaikan-cepat-prioritas-1-minggu}

\begin{enumerate}
\def\labelenumi{\arabic{enumi}.}
\tightlist
\item
  \textbf{Lengkapi UTS-4 (My SHAPE)} sesuai saran di atas;
\item
  \textbf{Tambahkan lirik + sinopsis} untuk tiap lagu di UTS-2 agar bisa
  dinilai isi/inspirasinya;
\item
  \textbf{Tulis 1 review personal penuh} (UTS-5) memakai rubrik (pilih
  salah satu karya Anda sendiri sebagai objek);
\item
  Revisi UTS-1 dengan pembuka anekdot + refleksi nilai diri.
\end{enumerate}

Jika Anda mau, saya bisa bantu merapikan UTS-4 (tabel SHAPE + narasi)
dan membuat kerangka cepat untuk \textbf{review} di UTS-5 dari salah
satu karya Anda.

\bookmarksetup{startatroot}

\chapter{UAS-1 My Concepts}\label{uas-1-my-concepts}

Mau hidup epik ? \href{lifestory.pdf}{Write your Life Story}

Apa itu berkonsep?

\url{https://youtu.be/QVfUlVBO80U?si=yM6q_rwV9rcDBbu7}

\bookmarksetup{startatroot}

\chapter{UAS-3 My Opinions}\label{uas-3-my-opinions}

SApa itu beropini? \href{BM\%20Opini.mp4}{Opini Berpengaruh}

Bagiamana menjaadi menarik? \href{./Interesting.mp4}{Menjadi Menarik}

\bookmarksetup{startatroot}

\chapter{UAS-3 My Innovations}\label{uas-3-my-innovations}

\bookmarksetup{startatroot}

\chapter{UAS-4 My Knowledge}\label{uas-4-my-knowledge}

Cara saya mengkomunikasikan sebuah pengatahuan sebagai petunjuk bagi
orang lain 1) saya tulis
\href{Rekomendasi\%20Presentasi\%20Efektif(Contoh\%20Makalah).pdf}{makalah
sebagai bahan utama} 2) lalu saya buat
\href{Contoh\%20Transkrip\%20Presentasi.pdf}{transkrip ucapan lisan} 3)
kemudian saya kembangkan
\href{Rekomendasi\%20Presentasi\%20(Contoh\%20Slides).pdf}{slide
pendukung trnsskrip} 4) lalu saya memproduksivideo audio visual
\url{https://youtu.be/ZbghfMvnPZc} \url{https://youtu.be/ZbghfMvnPZc}

\bookmarksetup{startatroot}

\chapter{UAS-5 My Professional
Reviews}\label{uas-5-my-professional-reviews}

Untuk melAkukan review, seperti pada
\href{../My_Personal_Reviews/Doc.5.Mengevaluasi-Esai-Berdasarkan-Rubrik.pdf}{pendekatan
AI}, kita membutuhkan rubrik

\bookmarksetup{startatroot}

\chapter{Summary}\label{summary}

In summary, this book has no content whatsoever.

\bookmarksetup{startatroot}

\chapter*{References}\label{references}
\addcontentsline{toc}{chapter}{References}

\markboth{References}{References}

\phantomsection\label{refs}




\end{document}
